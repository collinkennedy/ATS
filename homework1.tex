\documentclass[]{article}
\usepackage{lmodern}
\usepackage{amssymb,amsmath}
\usepackage{ifxetex,ifluatex}
\usepackage{fixltx2e} % provides \textsubscript
\ifnum 0\ifxetex 1\fi\ifluatex 1\fi=0 % if pdftex
  \usepackage[T1]{fontenc}
  \usepackage[utf8]{inputenc}
\else % if luatex or xelatex
  \ifxetex
    \usepackage{mathspec}
  \else
    \usepackage{fontspec}
  \fi
  \defaultfontfeatures{Ligatures=TeX,Scale=MatchLowercase}
\fi
% use upquote if available, for straight quotes in verbatim environments
\IfFileExists{upquote.sty}{\usepackage{upquote}}{}
% use microtype if available
\IfFileExists{microtype.sty}{%
\usepackage{microtype}
\UseMicrotypeSet[protrusion]{basicmath} % disable protrusion for tt fonts
}{}
\usepackage[margin=1in]{geometry}
\usepackage{hyperref}
\hypersetup{unicode=true,
            pdftitle={homework1},
            pdfauthor={Collin},
            pdfborder={0 0 0},
            breaklinks=true}
\urlstyle{same}  % don't use monospace font for urls
\usepackage{color}
\usepackage{fancyvrb}
\newcommand{\VerbBar}{|}
\newcommand{\VERB}{\Verb[commandchars=\\\{\}]}
\DefineVerbatimEnvironment{Highlighting}{Verbatim}{commandchars=\\\{\}}
% Add ',fontsize=\small' for more characters per line
\usepackage{framed}
\definecolor{shadecolor}{RGB}{248,248,248}
\newenvironment{Shaded}{\begin{snugshade}}{\end{snugshade}}
\newcommand{\AlertTok}[1]{\textcolor[rgb]{0.94,0.16,0.16}{#1}}
\newcommand{\AnnotationTok}[1]{\textcolor[rgb]{0.56,0.35,0.01}{\textbf{\textit{#1}}}}
\newcommand{\AttributeTok}[1]{\textcolor[rgb]{0.77,0.63,0.00}{#1}}
\newcommand{\BaseNTok}[1]{\textcolor[rgb]{0.00,0.00,0.81}{#1}}
\newcommand{\BuiltInTok}[1]{#1}
\newcommand{\CharTok}[1]{\textcolor[rgb]{0.31,0.60,0.02}{#1}}
\newcommand{\CommentTok}[1]{\textcolor[rgb]{0.56,0.35,0.01}{\textit{#1}}}
\newcommand{\CommentVarTok}[1]{\textcolor[rgb]{0.56,0.35,0.01}{\textbf{\textit{#1}}}}
\newcommand{\ConstantTok}[1]{\textcolor[rgb]{0.00,0.00,0.00}{#1}}
\newcommand{\ControlFlowTok}[1]{\textcolor[rgb]{0.13,0.29,0.53}{\textbf{#1}}}
\newcommand{\DataTypeTok}[1]{\textcolor[rgb]{0.13,0.29,0.53}{#1}}
\newcommand{\DecValTok}[1]{\textcolor[rgb]{0.00,0.00,0.81}{#1}}
\newcommand{\DocumentationTok}[1]{\textcolor[rgb]{0.56,0.35,0.01}{\textbf{\textit{#1}}}}
\newcommand{\ErrorTok}[1]{\textcolor[rgb]{0.64,0.00,0.00}{\textbf{#1}}}
\newcommand{\ExtensionTok}[1]{#1}
\newcommand{\FloatTok}[1]{\textcolor[rgb]{0.00,0.00,0.81}{#1}}
\newcommand{\FunctionTok}[1]{\textcolor[rgb]{0.00,0.00,0.00}{#1}}
\newcommand{\ImportTok}[1]{#1}
\newcommand{\InformationTok}[1]{\textcolor[rgb]{0.56,0.35,0.01}{\textbf{\textit{#1}}}}
\newcommand{\KeywordTok}[1]{\textcolor[rgb]{0.13,0.29,0.53}{\textbf{#1}}}
\newcommand{\NormalTok}[1]{#1}
\newcommand{\OperatorTok}[1]{\textcolor[rgb]{0.81,0.36,0.00}{\textbf{#1}}}
\newcommand{\OtherTok}[1]{\textcolor[rgb]{0.56,0.35,0.01}{#1}}
\newcommand{\PreprocessorTok}[1]{\textcolor[rgb]{0.56,0.35,0.01}{\textit{#1}}}
\newcommand{\RegionMarkerTok}[1]{#1}
\newcommand{\SpecialCharTok}[1]{\textcolor[rgb]{0.00,0.00,0.00}{#1}}
\newcommand{\SpecialStringTok}[1]{\textcolor[rgb]{0.31,0.60,0.02}{#1}}
\newcommand{\StringTok}[1]{\textcolor[rgb]{0.31,0.60,0.02}{#1}}
\newcommand{\VariableTok}[1]{\textcolor[rgb]{0.00,0.00,0.00}{#1}}
\newcommand{\VerbatimStringTok}[1]{\textcolor[rgb]{0.31,0.60,0.02}{#1}}
\newcommand{\WarningTok}[1]{\textcolor[rgb]{0.56,0.35,0.01}{\textbf{\textit{#1}}}}
\usepackage{graphicx,grffile}
\makeatletter
\def\maxwidth{\ifdim\Gin@nat@width>\linewidth\linewidth\else\Gin@nat@width\fi}
\def\maxheight{\ifdim\Gin@nat@height>\textheight\textheight\else\Gin@nat@height\fi}
\makeatother
% Scale images if necessary, so that they will not overflow the page
% margins by default, and it is still possible to overwrite the defaults
% using explicit options in \includegraphics[width, height, ...]{}
\setkeys{Gin}{width=\maxwidth,height=\maxheight,keepaspectratio}
\IfFileExists{parskip.sty}{%
\usepackage{parskip}
}{% else
\setlength{\parindent}{0pt}
\setlength{\parskip}{6pt plus 2pt minus 1pt}
}
\setlength{\emergencystretch}{3em}  % prevent overfull lines
\providecommand{\tightlist}{%
  \setlength{\itemsep}{0pt}\setlength{\parskip}{0pt}}
\setcounter{secnumdepth}{0}
% Redefines (sub)paragraphs to behave more like sections
\ifx\paragraph\undefined\else
\let\oldparagraph\paragraph
\renewcommand{\paragraph}[1]{\oldparagraph{#1}\mbox{}}
\fi
\ifx\subparagraph\undefined\else
\let\oldsubparagraph\subparagraph
\renewcommand{\subparagraph}[1]{\oldsubparagraph{#1}\mbox{}}
\fi

%%% Use protect on footnotes to avoid problems with footnotes in titles
\let\rmarkdownfootnote\footnote%
\def\footnote{\protect\rmarkdownfootnote}

%%% Change title format to be more compact
\usepackage{titling}

% Create subtitle command for use in maketitle
\providecommand{\subtitle}[1]{
  \posttitle{
    \begin{center}\large#1\end{center}
    }
}

\setlength{\droptitle}{-2em}

  \title{homework1}
    \pretitle{\vspace{\droptitle}\centering\huge}
  \posttitle{\par}
    \author{Collin}
    \preauthor{\centering\large\emph}
  \postauthor{\par}
      \predate{\centering\large\emph}
  \postdate{\par}
    \date{10/7/2020}


\begin{document}
\maketitle

\hypertarget{r-markdown}{%
\subsection{R Markdown}\label{r-markdown}}

This is an R Markdown document. Markdown is a simple formatting syntax
for authoring HTML, PDF, and MS Word documents. For more details on
using R Markdown see \url{http://rmarkdown.rstudio.com}.

\hypertarget{q1}{%
\subsubsection{Q1:}\label{q1}}

\emph{There are a number of seismic recordings from earthquakes and from
mining explosions in the R package ``astsa''. All of the data are in the
dataframe ``eqexp'', but two specific recordings are in ``EQ5'' and
``EX6'', the fifth earthquake and the sixth explosion, respectively. The
data represent two phases or arrivals along the surface, denoted by P (t
= 1,\ldots{}, 1024) and S (t = 1025,\ldots{}, 2048), at a seismic
recording station. The recording instruments are in Scandinavia and
monitor a Russian nuclear testing site. The general problem of interest
is in distinguishing between these waveforms in order to maintain a
comprehensive nuclear test ban treaty. To compare the earthquake and
explosion signals,}

\textbf{(a)} \emph{Plot the two series separately in a multifigure plot
with two rows and one column.}

\begin{verbatim}
## 
## Attaching package: 'aTSA'
\end{verbatim}

\begin{verbatim}
## The following object is masked from 'package:graphics':
## 
##     identify
\end{verbatim}

\begin{verbatim}
## 'data.frame':    2048 obs. of  17 variables:
##  $ EQ1: num  0.000818 0.033801 0.016628 0.004634 -0.036527 ...
##  $ EQ2: num  -0.208 0.0303 0.099 0.2034 0.5408 ...
##  $ EQ3: num  -0.021 -0.042 -0.0682 -0.0679 -0.0345 ...
##  $ EQ4: num  -0.0643 -0.0162 -0.0859 -0.0988 -0.0406 ...
##  $ EQ5: num  0.1347 0.087 0.1162 0.1134 0.0488 ...
##  $ EQ6: num  0.0509 0.077 0.0822 0.0588 0.0137 ...
##  $ EQ7: num  -0.0193 -0.02 -0.0115 -0.0259 -0.0261 ...
##  $ EQ8: num  -0.0219 -0.2527 -0.1716 -0.144 -0.3197 ...
##  $ EX1: num  0.04102 0.03004 0.02373 0.0129 0.00261 ...
##  $ EX2: num  0.0426 0.0264 0.0258 0.0337 0.0149 ...
##  $ EX3: num  -0.00805 -0.00192 -0.01116 -0.0183 -0.01309 ...
##  $ EX4: num  -0.073 -0.0947 -0.1144 -0.0908 -0.031 ...
##  $ EX5: num  -0.1106 -0.0982 -0.0781 -0.0642 -0.0482 ...
##  $ EX6: num  -0.02724 -0.00959 -0.03337 -0.00614 -0.01189 ...
##  $ EX7: num  -0.0563 -0.0851 -0.1113 -0.1039 -0.0826 ...
##  $ EX8: num  0.201 0.242 0.307 0.323 0.209 ...
##  $ NZ : num  0.3628 0.3382 0.0147 -0.3693 -0.3834 ...
\end{verbatim}

\includegraphics{homework1_files/figure-latex/plotdata-1.pdf}

\textbf{(b)} \emph{Plot the two series on the same graph using different
colors or different line types.}

\includegraphics{homework1_files/figure-latex/secondplot-1.pdf}

\textbf{(c)} \emph{Would you treat the earthquake and explosion series
as stationary or non-stationary? Support your answer.} Let's take a look
at the acf:

\begin{Shaded}
\begin{Highlighting}[]
\KeywordTok{acf}\NormalTok{(time_series_to_be_plotted) }\CommentTok{#Which do of these graphs do I analyze?}
\end{Highlighting}
\end{Shaded}

\includegraphics{homework1_files/figure-latex/acf-1.pdf}

Given the noticeable seasonality in the (individual) ACF plots of the
two time series, I believe the two time series are both \emph{not}
stationary

\textbf{(d)} \emph{Consider the signal plus noise model:
\includegraphics{https://canvas.ucdavis.edu/equation_images/x_t\%253Ds_t\%2520\%252B\%2520\%255Comega_t}
where
\includegraphics{https://canvas.ucdavis.edu/equation_images/\%255Comega_t\%255Csim\%2520WN(0\%252C1)}.}

\emph{Simulate and plot the following two models.} Model (i):

\includegraphics{https://canvas.ucdavis.edu/equation_images/x_t\%253Ds_t\%2520\%252B\%2520\%255Comega_t}
where
\includegraphics{https://canvas.ucdavis.edu/equation_images/\%255Comega_t\%2520\%255Csim\%2520WN(0\%252C1)}
\includegraphics{https://canvas.ucdavis.edu/equation_images/\%2520\%2520\%2520\%2520s_t\%2520\%253D\%2520\%250A\%2520\%2520\%2520\%2520\%2520\%2520\%255Cbegin\%257Bcases\%257D\%250A\%2520\%2520\%2520\%2520\%2520\%25200\%252C\%2520\%2526\%2520t\%253D1\%252C\%255Cdots\%252C125\%2520\%255C\%255C\%250A\%2520\%2520\%2520\%2520\%2520\%252010\%255C\%253Be\%255E\%257B-\%255Cfrac\%257Bt-125\%257D\%257B25\%257D\%257D\%255Ccos(2\%255Cpi\%2520t\%252F4)\%2520\%252C\%2526\%2520t\%253D126\%252C\%255Cdots\%252C250\%250A\%2520\%2520\%2520\%2520\%2520\%2520\%255Cend\%257Bcases\%257D}

Model (ii):
\includegraphics{https://canvas.ucdavis.edu/equation_images/x_t\%253Ds_t\%2520\%252B\%2520\%255Comega_t}
where
\includegraphics{https://canvas.ucdavis.edu/equation_images/\%255Comega_t\%2520\%255Csim\%2520WN(0\%252C1)}
\includegraphics{https://pi998nv7pc.execute-api.us-east-1.amazonaws.com/production/svg?tex=s_t\%20\%3D\%20\%0A\%5Cbegin\%7Bcases\%7D\%0A\%20\%200\%2C\%26\%20t\%3D1\%2C\%5Cdots\%2C125\%20\%5C\%5C\%0A\%20\%2010\%5C\%3Be\%5E\%7B-\%5Cfrac\%7Bt-125\%7D\%7B250\%7D\%7D\%5Ccos(2\%5Cpi\%20t\%2F4)\%20\%2C\%26\%20t\%3D126\%2C\%5Cdots\%2C250.\%0A\%5Cend\%7Bcases\%7D}

\emph{Compare the general appearance of the series of model (i) and (ii)
with the earthquake series and the explosion series.}

model i:

\begin{Shaded}
\begin{Highlighting}[]
\CommentTok{#model 1}
\NormalTok{s_fun1 =}\StringTok{ }\ControlFlowTok{function}\NormalTok{(t) }\KeywordTok{ifelse}\NormalTok{(t}\OperatorTok{<=}\DecValTok{125}\NormalTok{, }\DecValTok{0}\NormalTok{, }\DecValTok{10}\OperatorTok{*}\KeywordTok{exp}\NormalTok{(}\OperatorTok{-}\NormalTok{(t}\DecValTok{-125}\NormalTok{)}\OperatorTok{/}\DecValTok{25}\NormalTok{)}\OperatorTok{*}\KeywordTok{cos}\NormalTok{(}\DecValTok{2}\OperatorTok{*}\NormalTok{pi}\OperatorTok{*}\NormalTok{t}\OperatorTok{/}\DecValTok{4}\NormalTok{))}
\NormalTok{omega_fun1 =}\StringTok{ }\ControlFlowTok{function}\NormalTok{(t) }\KeywordTok{rnorm}\NormalTok{(}\KeywordTok{length}\NormalTok{(t))}

\NormalTok{t_index =}\StringTok{ }\DecValTok{1}\OperatorTok{:}\DecValTok{250}
\KeywordTok{set.seed}\NormalTok{(}\DecValTok{19460614}\NormalTok{)}
\NormalTok{x1 =}\StringTok{ }\KeywordTok{ts}\NormalTok{(}\KeywordTok{s_fun1}\NormalTok{(t_index) }\OperatorTok{+}\StringTok{ }\KeywordTok{omega_fun1}\NormalTok{(t_index))}

\KeywordTok{str}\NormalTok{(x1) }\CommentTok{#str function tells us the object type of the argument (x)}
\end{Highlighting}
\end{Shaded}

\begin{verbatim}
##  Time-Series [1:250] from 1 to 250: 0.13 -0.383 -0.192 0.515 0.441 ...
\end{verbatim}

model ii:

\begin{Shaded}
\begin{Highlighting}[]
\CommentTok{#model 2}
\NormalTok{s_fun2 =}\StringTok{ }\ControlFlowTok{function}\NormalTok{(t) }\KeywordTok{ifelse}\NormalTok{(t}\OperatorTok{<=}\DecValTok{125}\NormalTok{, }\DecValTok{0}\NormalTok{, }\DecValTok{10}\OperatorTok{*}\KeywordTok{exp}\NormalTok{(}\OperatorTok{-}\NormalTok{(t}\DecValTok{-125}\NormalTok{)}\OperatorTok{/}\DecValTok{250}\NormalTok{)}\OperatorTok{*}\KeywordTok{cos}\NormalTok{(}\DecValTok{2}\OperatorTok{*}\NormalTok{pi}\OperatorTok{*}\NormalTok{t}\OperatorTok{/}\DecValTok{4}\NormalTok{))}
\NormalTok{omega_fun2 =}\StringTok{ }\ControlFlowTok{function}\NormalTok{(t) }\KeywordTok{rnorm}\NormalTok{(}\KeywordTok{length}\NormalTok{(t))}

\NormalTok{t_index =}\StringTok{ }\DecValTok{1}\OperatorTok{:}\DecValTok{250}
\KeywordTok{set.seed}\NormalTok{(}\DecValTok{19460614}\NormalTok{)}
\NormalTok{x2 =}\StringTok{ }\KeywordTok{ts}\NormalTok{(}\KeywordTok{s_fun2}\NormalTok{(t_index) }\OperatorTok{+}\StringTok{ }\KeywordTok{omega_fun2}\NormalTok{(t_index))}

\KeywordTok{str}\NormalTok{(x2) }\CommentTok{#str function tells us the object type of the argument (x)}
\end{Highlighting}
\end{Shaded}

\begin{verbatim}
##  Time-Series [1:250] from 1 to 250: 0.13 -0.383 -0.192 0.515 0.441 ...
\end{verbatim}

Plotting them both:

\begin{Shaded}
\begin{Highlighting}[]
\KeywordTok{plot.ts}\NormalTok{(x1, }\DataTypeTok{main =} \StringTok{"model i"}\NormalTok{)}
\end{Highlighting}
\end{Shaded}

\includegraphics{homework1_files/figure-latex/plottingmodeliandmodelii-1.pdf}

\begin{Shaded}
\begin{Highlighting}[]
\KeywordTok{plot.ts}\NormalTok{(x2, }\DataTypeTok{main =} \StringTok{"model ii"}\NormalTok{)}
\end{Highlighting}
\end{Shaded}

\includegraphics{homework1_files/figure-latex/plottingmodeliandmodelii-2.pdf}

\begin{Shaded}
\begin{Highlighting}[]
\KeywordTok{plot.ts}\NormalTok{(time_series_to_be_plotted,}
        \DataTypeTok{plot.type=}\StringTok{"single"}\NormalTok{,}
        \DataTypeTok{main=}\StringTok{"P & S Waves- Earthquake and Explosion Series"}\NormalTok{,}
        \DataTypeTok{col=}\NormalTok{colors_name, }\CommentTok{#associate red with column EQ5, blue with column EX6}
        \DataTypeTok{ylab=}\StringTok{"units"}\NormalTok{)}
\KeywordTok{legend}\NormalTok{(}\StringTok{"bottomleft"}\NormalTok{,}
\NormalTok{       time_series_name,}
       \DataTypeTok{col=}\NormalTok{colors_name,}
       \DataTypeTok{lty=}\DecValTok{1}\NormalTok{)}
\end{Highlighting}
\end{Shaded}

\includegraphics{homework1_files/figure-latex/plottingmodeliandmodelii-3.pdf}

Notice that in the case of model i and model ii, the spike in signal
occurs later in the time series (about halfway along the time series).
This is contrasted against the spike seen in the Explosion Series, where
a spike occurs near the beginning, and then again later, also about
halfway through the series.

all the of the time series (model 1, model 2, Earthquake \& Explosion)
are centered about 0.

Also, the Earthquake series experiences a spike in () level, and it
stays at that higher level for the remainder of the time series,
similarly to the pattern seen in model 2.


\end{document}
